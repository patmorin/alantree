\documentclass{patmorin}
\usepackage{pat}


\begin{document}
\section{The Exponential Process (First Passage Percolation)}

Consider the following process for growing a spanning tree, $T$, of an
$n$-vertex connected undirected graph $G$ starting from a distinguished
vertex $s\in V(G)$.  On each edge of $G$ we attach an exponential timer
with rate 1. When the timer on an edge $vw$ rings the timer is immediately
reset and, if exactly one of $v$ or $w$ is in $T$, then the edge $vw$
is added to $T$.  We call this Process~E.

\begin{lem}\lemlabel{percolation}
  If $G$ has diameter $D$ then, with probability at least $1-1/n$, $T$
  is a spanning tree after time at most $c(D+\log n)$.
\end{lem}


\begin{proof}
  Let $v$ be a vertex of $G$ such that there exists a path
  $v_0,\ldots,v_k$ of length $k$ in $G$ from $s=v_0$ to $v=v_k$.
  Let $e_i=v_{i-1}v_i$ be the $i$th edge on this path.

  For each $i\in\{1,\ldots,k\}$, the vertex $v_i$ joins the tree at some
  time $t_i$.  Once this happens, the vertex $v_{i+1}$ will join the tree
  at or before the next ring of the timer on edge $e_{i+1}$.  (Note that
  $v_{i+1}$ may join the tree before this ring and even before $t_i$.)
  By the memoryless property of exponentials, this next ring occurs at
  time $t_i+Z_i$ where $Z_i$ is a an exponential random variable with
  rate 1.  Therefore, the random variable $t_k$ is dominated by the sum
  of $k$ independent exponentials $Z_1,\ldots,Z_k$.
  In particular, using standard concentration results,
  \[
      \Pr\{t_k\ge c(k+\log n)\} \le \Pr\{Z_1+\cdots Z_k\ge c(k+\log n)\} \le 1/n^2 \enspace .
  \]
  For each $v\in V(G)$, let $t(v)$ denote the time at which $v$ joins
  the tree $T$ and define $t=\max\{t(v):v\in V(G)\}$
  as the time at which the last vertex of $G$ joins 
  $T$.  Then,
  \[
      \Pr\{t> c(D+\log n)\} \le \sum_{v\in V(G)}\Pr\{t(v) \ge c(D+\log n)     
       \le 1/n \enspace . \qedhere
  \]
\end{proof}

Now note that the tree, $T$, that results from the preceding process
is distributed in the same way as the spanning tree that one gets by
starting with the tree $T=(\{s\},\emptyset)$ and repeatedly choosing
a uniform random edge $vw$ and adding $vw$ to $T$ if and only if $vw$
has exactly one endpoint in $V(T)$.  We call this Process~A.

\begin{lem}
  If $G$ has diameter $D$ and maximum degree $\Delta$ then, with
  probability at least $1-1/n$, the spanning tree $T$ has diameter at
  most $c\Delta(D+\log n)$.
\end{lem}

\begin{proof}
  Let $P=v_0,v_1,\ldots,v_L$ be some simple path in $G$ starting at
  $s=v_0$ and let $e_i=v_{i-1}v_i$ be the $i$th edge on this path.
  In the model of Process~E, this path can not appear before the
  timer on $e_1$ rings, and then the timer on $e_2$ rings, and so on,
  until the timer on $e_L$ rings.  Let $t_L$ be the time at which this
  first occurs.  Then $t_L$ is the sum of $L$ independent exponential(1)
  random variables and
  \[
      \Pr\{t_L\} \le L/\Delta \le \exp(-\alpha \Delta L) \enspace .
  \]
  If $P$ appears in $T$, then either $t_L < c(D+\log n)$ or $T$ is not
  yet a spanning tree of $G$ at time $c(D+\log n)$.  
  Therefore, 
  \[
      \Pr\{\text{$P$ appears in $T$}\} \le 1/n + \exp(-\alpha L) \enspace .
  \]
  Let $P_L$ be the set of all simple paths in $G$ that start at $s$ and have length $L$.  The probability that $T$ contains any path from $P_L$ is at most
  \[
      1/n + |P_L|\exp(-\alpha L) \le 1/n + (\Delta-1)^L\exp(-\alpha L)
  \]
  Therefore, the probability that $T$ contains any simple path that
  starts at $s$ and contain 


Let $\mathcal{P}$ be
  the set of all simple paths in $G$ originating from $s$ and having length $L$.


Therefore, the probability
  that any such path $P$


  to upper bound the probability that this path appears in $T$.  
\end{proof}


\end{document}
