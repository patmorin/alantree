\documentclass{patmorin}
\usepackage{pat}


\begin{document}
\section{Inequalities for Sums of Exponentials}

We will make use of two inequalities on sums of exponential random
variables, both of which can be obtained using Chernoff's bounding method.
If $Z_1,\ldots,Z_k$ are independent exponential random variables, each
with mean $\lambda=1$, then for all $d>1$,
\begin{equation}
    \Pr\left\{\sum_{i=1}^k Z_i \le k/d\right\} \le \exp(-k(\ln d -1 + 1/d)) \le \left(\frac{e}{d}\right)^k  \eqlabel{head-bound}
\end{equation}
and for all $t>1$, 
\begin{equation}
    \Pr\left\{\sum_{i=1}^n Z_i \ge tk\right\} \le \exp (k- k t/2) \enspace . \eqlabel{tail-bound}
\end{equation}


\section{The Exponential Process (First Passage Percolation)}

Consider the following process for growing a spanning tree, $T$, of an
$n$-vertex connected undirected graph $G$ starting from a distinguished
vertex $s\in V(G)$.  On each edge of $G$ we attach an exponential timer
with rate 1. When the timer on an edge $vw$ rings the timer is immediately
reset and, if exactly one of $v$ or $w$ is in $T$, then the edge $vw$
is added to $T$.  We call this \emph{Process~E} and we say that Process~E
is \emph{complete} once $T$ spans $G$.  Note that Process~E is equivalent to selecting an exponential(1) edge weight for each edge of $G$ and then computing the shortest path tree rooted at $s$.  We call this latter process Process~FP, for first-passage percolation.

\begin{lem}\lemlabel{percolation}
  If $G$ has diameter $D$ then, with probability at least $1-1/n$, $T$
  is a spanning tree after time at most $c(D+\log n)$.
\end{lem}


\begin{proof}
  Let $v$ be a vertex of $G$ such that there exists a path
  $v_0,\ldots,v_k$ of length $k$ in $G$ from $s=v_0$ to $v=v_k$.
  Let $e_i=v_{i-1}v_i$ be the $i$th edge on this path.

  For each $i\in\{1,\ldots,k\}$, the vertex $v_i$ joins the tree at some
  time $t_i$.  Once this happens, the vertex $v_{i+1}$ will join the tree
  at or before the next ring of the timer on edge $e_{i+1}$.  (Note that
  $v_{i+1}$ may join the tree before this ring and even before $t_i$.)
  By the memoryless property of exponentials, this next ring occurs at
  time $t_i+Z_i$ where $Z_i$ is a an exponential random variable with
  rate 1.  Therefore, the random variable $t_k$ is dominated by the sum
  of $k$ independent exponentials $Z_1,\ldots,Z_k$.
  In particular, using \eqref{tail-bound},
  \[
      \Pr\{t_k\ge c(k+\log n)\} \le \Pr\{Z_1+\cdots Z_k\ge c(k+\log n)\} \le 1/n^2 \enspace .
  \]
  For each $v\in V(G)$, let $t(v)$ denote the time at which $v$ joins
  the tree $T$ and define $t=\max\{t(v):v\in V(G)\}$
  as the time at which the last vertex of $G$ joins 
  $T$.  Then,
  \[
      \Pr\{t> c(D+\log n)\} \le \sum_{v\in V(G)}\Pr\{t(v) \ge c(D+\log n)     
       \le 1/n \enspace . \qedhere
  \]
\end{proof}

\section{The Random Edge Choice Process}

Now note that the tree, $T$, that results from the preceding process
is distributed in the same way as the spanning tree that one gets by
starting with the tree $T=(\{s\},\emptyset)$ and repeatedly choosing
a uniform random edge $vw$ and adding $vw$ to $T$ if and only if $vw$
has exactly one endpoint in $V(T)$.  We the call second version Process~B.
That Process~E and Process~B produce identically distributed spanning
trees follows from the memorylessness of exponential random variables.
In particular, at any point in time, each of the $|E(G)|$ timers in
Process~E is equally likely to be the next timer to ring.

\subsection{The Upper Bound}

\begin{lem}\lemlabel{alantree-upper-bound}
  If $G$ has diameter $D$ and maximum degree $\Delta$ then after running
  Process~B to completion, with probability at least $1-O(1/n)$, the spanning
  tree $T$ has diameter at most $2c\Delta(D+\log n)$.
\end{lem}

\begin{proof}
  Let $P=v_0,v_1,\ldots,v_L$ be some simple path in $G$ starting at
  $s=v_0$ and let $e_i=v_{i-1}v_i$ be the $i$th edge on this path.
  In the model of Process~E, this path can not appear before the timer
  on $e_1$ rings, and then the timer on $e_2$ rings, and so on, until the
  timer on $e_L$ rings.  Let $t_L$ be the time at which this first occurs.
  Then $t_L$ is lower-bounded by the sum of $L$ independent exponential(1)
  random variables.  Using \eqref{head-bound}, we obtain
  \[
      \Pr\{t_L\} \le L/d \le \left(\frac{e}{d}\right)^L \enspace .
  \]
  If $P$ appears in $T$, then either $t_L < c(D+\log n)$ or $T$ is not
  yet a spanning tree of $G$ at time $c(D+\log n)$.  
  Therefore by the preceding discussion and \lemref{percolation}, 
  \[
      \Pr\{\text{$P$ appears in $T$}\} \le 1/n 
        + \left(\frac{ec(D+\log n)}{L}\right)^L \enspace .
  \]
  Let $P_L$ be the set of all simple paths in $G$ that start at $s$ and have length $L$.  The probability that $T$ contains any path from $P_L$ is at most
  \[
      1/n + |P_L|\left(\frac{\Delta ec(D+\log n)}{L}\right)^L 
      < 1/n + \left(\frac{\Delta ec(D+\log n)}{L}\right)^L \enspace ,
  \]
  since $|P_L|\le \Delta^L$.
  Finally, the probability that $T$ contains any path of length at least
  $L'=2\Delta ec(D+\log n)$ is at most
  \[
     \frac{1}{n}+\sum_{L=L'}^{n-1} \left(\frac{\Delta ec(D+\log n)}{L}\right)^L <  \frac{1}{n}+\sum_{L=L'}^{n-1} (1/2)^L < 2/n \enspace . \qedhere
  \]
\end{proof}

I believe the dependence on $\Delta$ is far from optimal and that $\log\Delta$ is the truth.  The main slack in the argument comes from the fact that, even if the timers on $e_1,\ldots,e_L$ ring in the required amount of time, this doesn't mean that the path $P$ appears in $G$.  Indeed, any of the vertices in this path could have joined $T$ sooner, in which case the path won't appear. 


\subsection{The Lower Bound}

\begin{lem}
  For every $D,\Delta\in\N$, there exists a graph $G$ with diameter
  $D$ and maximum degree $\Delta$ such that after running Process~B
  to completion, the expected height of $T$ is $\Omega(\Delta D)$.
\end{lem}

\begin{proof}
  The graph $G$ is constructed as follows.  Start with a vertex $s$ and
  $L$ groups $V_1,\ldots,V_L$ each containing $\delta$ vertices. There
  is an edge joining $s$ to every vertex in $V_1$ and, for each
  $i\in\{1,\ldots,L-1\}$, every vertex in $V_i$ is adjacent to every
  vertex in $V_{i+1}$.  Call this graph $G^-$.

  Next, we augment $G^-$ by selecting one evertex $v_i$ from each group
  $V_i$ and building a complete binary tree, $T_G$ whose leaves, in order,
  are $v_1,\ldots,v_L$.  Finally, we subdivide the edges of $T_G$ as follows:
  If $v$ is a node of height $h\ge 1$ in $T_G$, then we subdivide the two
  edges leading to $v$'s children into paths of length
  \[
      s(h) = a\cdot\max\{\log L, 2^h/\delta\} \enspace .
  \]
  Call the resulting tree $T_G'$.  The graph $G$ is the union of $G^-$ and $T_G'$.

  Now, between any two vertices in $G$ there is path all but two of
  whose edges are in $T_G'$ and of length at most
  \[
      D = 2+\sum_{h=1}^{\log L} s(h) = O(L/\Delta) \enspace .
  \]
  TODO: Do this calculation more carefully.

  Now, imagine runing Process~FP on $G$.  First consider the subgraph
  $G^-$.  Starting at $s$ and taking the lowest weight edge leads us to
  some vertex in $V_1$.  From there we can take the lowest-weight edge to
  reach a vertex in $V_2$, and so on, until reaching $V_L$.  The weight
  of the resulting path is the sum of $L$ $\exponential(1/\delta)$
  random variables.  For $t>1$, the probability that the weight of
  this path exceeds $(b+1)L/\delta$ is therefore at most $\exp(-bL)$.
  Therefore, with probability at least $1-\exp(-tL)$, the shortest path
  in $P$ in $G^-$ from $s$ to any vertex $t$ in $V_L$ has weight at
  most $(b+1)L/\Delta$.

  The path $P$ will be will appear in the tree $T$ generated by Process~FP
  unless there are are $i,j \in\binom{[L]}{2}$ such that the tree $T_G'$
  contains a path from $v_i$ to $v_j$ of weight less than


  


  groups   
  The graph $G$ is constructed as follows: Start with a complete
  $\delta$-ary tree, $T_G$ of height $L=aD\Delta$ for some constant $a$
  and some $\delta < \Delta$, each specified later.  Next, for each
  root-to-leaf path $P$ in $T_G$, create a complete binary tree $T_P$
  whose leaves are the vertices of $P$ in the same order they appear
  in $P$.  Finally, subdivide each edge $e$ of $T_P$  $L/(\Delta \log_2
  L)$ times so that $e$ becomes a path of length $L/(\Delta\log_2 L)$.

  We claim that the resulting graph, $G$, has diameter $D\in O(L/\Delta)$.
  (This isn't hard to show; For any two vertices in $T_G$, route to
  their lowest common ancestor in $T_G$ using the two trees attached
  to root-to-leaf paths.)  

  Note that, as described above, the graph $G$ would have maximum degree
  $\delta+\delta_L$, since the root of $T_G$ takes part in $\delta^L$
  root-to-leaf paths.  To prevent this, we identify pairs of vertices in
  these trees as follows:  Each vertex of $T_P$ corresponds to a segment
  of root-to-leaf path in $T_G$.  If two vertices $v$ and $w$ from two
  different trees $T_P$ and $T_Q$ correspond to the same segment, then
  we identify $v$ and $w$.  We claim that the resulting graph has maximum
  degree $\delta^2$. (This is either easy to prove, or not true.)

  Let $s$ be the root of $T_G$.  Now, assign exponential(1) edge weights
  to the edges of $G$.  First focus on a special subtree of $T_G$
  obtained by starting at $s$ keeping only each child $w$ of $s$ in $T_G$
  such that the edge weight on $sw$ is at most $c/\delta$ for some
  constant $c$ discussed later.  We then do this recursively for each
  such child $w$ of $s$ that survives.
 
  The tree $T'_G$ that results from this is a truncated Galton-Watson
  tree in which the expected number of children at any node is (slightly
  greater than) $c$.  Therefore, with probability at least $p$ (insert
  a constant here---$p=1/2$?), $T'_G$ has at least $c^{L}$ leaves.

  Now, consider one of the root-to-leaf paths, $P$, in $T'_G$.  Attached
  to this path is a (subdivided) tree $T_P$.  Each vertex $v$ of $T_P$ is
  the root of a subtree of height $h$ whose leaves are a subpath of $P$
  of length $2^h$.  (Here the height $h$ is the height of $v$ in $T_P$
  before subdividing edges.)  Now $v$ is incident to two edges $e_1$
  and and $e_2$ leading to its two children in $T_P$.  After subdividing,
  $e_1$ and $e_2$ become paths of length $L/(\Delta\log_2 L)$.
  We say that $v$ is \emph{unlucky} if the sum of edge weights on each
  of these paths exceeds $2^h/\delta$. 

  The probability that one of these sums exceeds $2^h/\delta$ is the probability
  that the sum of $k=L/(\delta\log_2(L)$ independent exponential(1) random variables exceeds $2^{h}/\Delta$:
  \begin{align*}
      \Pr\{\sum_{i=1}^{k} X_i \ge 2^{h}/\Delta\}
          & \approx \exp(-k (k/-k^2\delta/2^{h})
          & = \exp(-L^2/(\delta 2^{h}\log^2 L))
  \end{align}


$c and, after subdivding, 
+

branching process defineds on $T_G$ as follows.  Start at $s$ and make
  it adadd each
  every child of $
the root 

two vertices $v$ and $w$ of two such trees $T_P$ and $T_Q$ if 


 
  To see this, note the each tree $T_P$ has height $\log_2(L)$, but each
  edge of $T_P$ was subdivided $L/(\Delta\log_2 L)$ times.  Therefore,
  the graph induced by the vertices of $T_P$ has diameter $O(L/\Delta)$.

  Now, for 

, so it has height
  $L/(\Delta \log_2 D)$

  Notice that the resulting graph has diameter $L/\Delta$


  The graph $G$ consists of a $\delta$-ary tree for some $\delta = D-2$.



  The graph $G$ is constructed by starting with a complete $b$-ary tree
  $U$ of height $h$.  Next we add a vertex $w$ and a separate path of
  length $k$ from $w$ to each vertex of $T$.  Call the resulting graph
  $G$ an observe that the average degree of $G$ is $\Delta =O(b)$ and
  that the diameter of $G$ is at most $D=2k$.

  Consider the \emph{greedy path} in $U$ that starts at $s$ and
  proceeds to the first child, $u$, of $s$ whose timer rings, and then
  to the first child of $u$ whose timer rings, and so on until reaching
  a leaf of $U$. Let $t_1$ denote the time at which the greedy path
  reaches a leaf of $U$ and note that the greedy path will appear in
  the tree $T$ generated by Process~E at time $t_1$ unless $w$ joins
  $T$ before time $t_1$ (in which case the greedy path may
  or may not appear in $T$).

  The time $t_1$ is the sum of independent random variables
  $Z_1,\ldots,Z_h$, where each $Z_i$ is the minimum of $d$ exponential(1)
  random variables, so each $Z_i$ is an exponential with mean $1/b$.
  The mean of $t_1$ is therefore $h/b$ and, for any $a>1$, with
  probability at least $1-\exp((1-a)h)$,  $t_1\le ah/b$.\footnote{Check
  this last tail bound.}

  We now study the time at which $w$ joins $T$.  Notice that this happens
  through some path that starts at $s$, walks in $U$ until some vertex $u$,
  and then walks $k$ more steps outside of $U$ from $u$ to $w$.  When this
  happens we say that \emph{$w$ joins $T$ through $u$}.
  Fix a vertex $u$ at depth $r$ in $U$.  The probability that $w$ joins
  $T$ through $u$ by time $ah/b$ is at most the probability that the $r+k$
  relevant exponential(1) timers $Z_1,\ldots,Z_{r+k}$ sum to at most $ah/b$:
  \[
  \Pr\{\text{$w$ joins $T$ through $u$ by time $ah/b$}\} \le 
  \Pr\left\{\sum_{i=1}^{r+k} Z_k \le ah/b\right\} = \left(\frac{eah}{b(r+k)}\right)^{r+k} \enspace .
  \]
  The tree, $U$ has $b^r$ vertices of depth $r$, so 
  \begin{align*}
   \Pr\{\text{$w$ joins $T$ by time $ah/b$}\} 
      & \le \sum_{r=0}^h b^r \left(\frac{eah}{b(r+k)}\right)^{r+k} \\
      & == \sum_{r=0}^h b^{-k} \left(\frac{eah}{r+k}\right)^{r+k} 
    \enspace \ldots
  \end{align*}
  CRAP!
\end{proof}


Notice that, except for $w$, every vertex in the graph $G$ created in
the preceding proof has degree at most $b+1$.  
\end{document}
