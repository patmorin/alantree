\documentclass{article}
\usepackage[utf8]{inputenc}
\usepackage{amsmath,url,color}
\begin{document}

\section{Hypercubes}
Consider the $d$-dimensional hypercube,
with $2^d$ vertices and $m=d2^{d-1}$ edges, and suppose $s$ is an arbitrary vertex.

Process 1:
We grow a tree.
Initially, the tree has just one vertex, $s$.
In each round, we choose a random edge among all edges with exactly one endpoint in the tree, and add it to tree.
We stop when the tree spans the graph.

Theorem.
Whp depth of this tree is $O(d)$.

Process 2:
We grow a tree, starting with vertex $s$ initially.
In each round, we choose a uniformly random edge from the graph.
If the edge has exactly one endpoint in the tree, we add it to the tree.
We stop when the tree spans the graph.

Clearly, the tree generated by Process 2 has the same distribution as that created by Process 1
(Process 2 is a slower process, but this is not important for us);
so we will work with Process 2.

Lemma (proof appears further down). There exists a constant $\eta<30$ such that
whp Process 2 is completed after at most $\eta m$ many rounds.

We prove the theorem now.
Run Process 2 until it is completed.
By Lemma we may assume this happens after at most $\eta m$ rounds.
Fix a path of length $L$ starting from $s$. We can upper bound the probability that this path appears in the tree by
$$
\binom{\eta m}{L} \left( \frac{1}{m} \right)^L
\leq \left( \frac{e \eta m}{mL} \right)^L \enspace .
$$
In the LHS here,
$\binom{\eta m}{L}$
is the number of ways to choose $L$ specific rounds for the $L$ edges of the path to be chosen,
and for each of these rounds, the probability to choose a specific edge is $1/m$.
There are $d^L$ possible such paths, so the probability that any such path of length $L$ appears is upper bounded by
$$
d^L
\left( \frac{e \eta m}{mL} \right)^L
= \left( \frac{e \eta d m}{mL} \right)^L,
$$
which is $o(1)$ for $L>e \eta d$, as required.

Proof of Lemma.
We use Corollary 6.3 from
Fill and Pemantle,
Percolation, first-passage percolation and covering
times for Richardson's model on the $n$-cube,
The Annals of Applied Probability 1993.
To state this result, we need to define another process.

Process 3:
We grow a tree, starting with vertex $s$ initially.
Each edge is equipped with an exponential clock with rate 1.
Whenever the clock of an edge rings, if the edge has exactly one endpoint in the tree, the edge is added to the tree.

Corollary 6.3 says that whp the time it takes for the tree to span the graph is at most $\eta <15$.

Notice that, the time between two clock rings is an exponential with rate $m$,
and by memorylessness of the exponentials, whenever we hear \emph{some} ring, the ring is equally likely to come from any edge.
This gives a natural coupling between Process 3 and Process 2.
Indeed, let $Z_1,Z_2,\dots$ be independent exponentials with rate $m$, and let $R$ be the number of rounds for Process 2 to finish.
Then, Process 3 is finished at time precisely
$Z_1 + \dots + Z_R$.
Therefore,
\begin{align*}
\mathbf{Pr} [ R > 2 \eta m ] & \leq 
\mathbf{Pr} [Z_1 + \dots + Z_R > \eta]+
\mathbf{Pr} [Z_1 + \dots + Z_{2\eta m} \leq \eta]
\leq o(1) + o(1),
\end{align*}
where for the last inequality we used Corollary 6.3,
and the fact that
$\mathbf{E}[Z_1 + \dots + Z_{2\eta m}]= 2\eta$,
and that 
$Z_1 + \dots + Z_{2\eta m}$ is sharply concentrated around its mean.

Remark 1:  
The upper bound on  $\eta$ was improved to 1.574
in
Martinsson, Anders. Unoriented first-passage percolation on the n-cube. Ann. Appl. Probab. 26 (2016), no. 5, 2597--2625. MR3563188.

Remark 2: I guess Theorem should be true with probability $1-e^{-\Omega(d)}$. One needs to check the confidence guarantee of Corollary 6.3.

\section{General graphs}

Now consider a general graph on $n$ vertices
with max degree $\Delta$ and diameter $D$.
We want to show the tree generated by Process 1
has depth $O(\Delta (D + \log n))$.
We first show that 
Process 3 covers the graph within time $O(D+\log n)$,
which is almost like saying 
Process 2 covers the graph within $O(m D+m\log n)$ rounds.

For proving this we will use the following concentration inequality, whose proof uses standard moment-generating-function argument, 
and follows from Theorem~5.1(i) in
\url{http://www2.math.uu.se/~svante/papers/sjN14.pdf}:\\
Let $X_1,X_2,\dots$ be i.i.d.\ exponentials with rate 1.
For any $t>1$,
$$\mathbf{Pr}[X_1+\dots+X_k > kt] < \exp (k- k t/2)  $$

Now, suppose the tree starts from $s$,
and we will show that the probability that Process 3 does not cover a given vertex $v$ by time $4 (D + \log n)$, is $< 1/n^2$.
Consider a path of length $k \leq D$ between $s$ and $v$. Then we have
\begin{align*}
\mathbf{Pr}[v \textnormal{ not covered by time }4 (D + \log n)] &< 
\mathbf{Pr}[X_1+\dots+X_k > 4(D + \log n)] \\
&<\exp(k - 2D - 2 \log n) < \exp(-2 \log n).
\end{align*}
\\ \framebox{
\begin{minipage}{\textwidth}
Pat says: I think the preceding needs more justification. Suggest: For a path, $s=v_0,v_1,\ldots,v_k=v$, the vertex $v_1$ will join the the tree by the time timer on $v_0v_1$ rings. More generally, for $i\ge 1$, if $v_{i-1}$ joins the tree at time $t$,
then $v_i$ will join the tree at or before the first ring of the timer on $v_{i-1}v_i$ that
occurs after time $t$. (Note that $v_i$ may join the tree sooner than this (even before $v_{i-1}$) through some other path from $s$.)
\end{minipage}
}

We now prove the tree generated by Process 2
has depth $O(\Delta (D + \log n))$.
Run Process 2 until it is completed.
By above calculations we may assume this happens after at most $4 (D + \log n) m$ rounds.
Fix a path of length $L$ starting from $s$. We can upper bound the probability that any such path appears in the tree by

$$
\Delta^L
\binom{4 (D + \log n) m}{L} \left( \frac{1}{m} \right)^L
\leq \left( \frac{ 4e \Delta(D + \log n)  m}{mL} \right)^L,
$$
which is $o(1)$ for $L>4e \Delta(D + \log n)$, as required.
In the LHS here,
$\Delta^L$ is the number of possible paths,
$\binom{4 (D + \log n) m}{L}$
is the number of ways to choose $L$ specific rounds for these $L$ edges to be chosen,
and for each of these rounds, the probability to choose a specific edge is $1/m$.
\\ \framebox{
\begin{minipage}{\textwidth}
Pat says: I wonder if the dependence on $\Delta$ is optimal. The example of a sequence of $D$ $(\Delta-1)$-cliques joined in path gives a lower bound of $\Omega(D\log\Delta)$. 
\end{minipage}
}



\end{document}

